\documentclass[12pt]{article}

% Packages for math and formatting
\usepackage{amsmath}
\usepackage{amssymb}
\usepackage{amsfonts}

% Document settings
\setlength{\parindent}{0pt}
\setlength{\parskip}{1em}
\setlength{\textheight}{22cm}
\setlength{\textwidth}{15cm}
\setlength{\topmargin}{-1cm}
\setlength{\oddsidemargin}{0.5cm}
\setlength{\evensidemargin}{0.5cm}

\begin{document}

\begin{center}
    \Large \textbf{First Principle of Finite Induction} \\
    \vspace{0.5cm}
    \large Ernest Michael Nelson \\
    \vspace{0.5cm}
    \normalsize December 15, 2024
\end{center}

\vspace{1.5cm}

\section{Introduction}

In this paper, we will be doing a step-by-step solution to a common induction problem. Also, it is shown and given in many books in the study of number theory and reasoning and proofs.

\section{Formula}

We are given the formula
\[
\sum_{n=1}^\infty n^2 = \frac{n(2n+1)(n+1)}{6}
\]
with \( n \) being an element of the natural numbers.

\section{Proof:}

Now we begin by running off a few terms to help see the pattern emerge.

\[
\sum_{n=1}^\infty 1^2 + 2^2 + 3^2 + 4^2 + 5^2 + \cdots + n^2 = \frac{n(2n+1)(n+1)}{6}
\]

Let's assume that \( n=1 \).

\[
1 = \frac{1(2(1)+1)(1+1)}{6} = 1
\]
\[
1 = \frac{(2+1)(1+1)}{6} = 1
\]
\[
1 = \frac{6}{6} = 1
\]

Next, we assume that \( n = k \) and that \( k \) is an element of the natural numbers. An equation we will use in the induction hypothesis is denoted as Equation 1.

\[
\sum_{k=1}^\infty 1^2 + 2^2 + 3^2 + 4^2 + 5^2 + \cdots + k^2 = \frac{k(2k+1)(k+1)}{6}
\]

\end{document}
